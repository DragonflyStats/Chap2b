% Identity Plot
% Variations of the Bland Altman Plot
% Survival Plot (Luiz et al)
% Bartko's Ellipse
% Mountain Plot

\documentclass[12pt, a4paper]{article}
\usepackage{natbib}
\usepackage{vmargin}
\usepackage{graphicx}
\usepackage{epsfig}
\usepackage{subfigure}
%\usepackage{amscd}
\usepackage{amssymb}
\usepackage{amsbsy}
\usepackage{amsthm, amsmath}
%\usepackage[dvips]{graphicx}
\bibliographystyle{chicago}
\renewcommand{\baselinestretch}{1.8}

% left top textwidth textheight headheight % headsep footheight footskip
\setmargins{3.0cm}{2.5cm}{15.5 cm}{23.5cm}{0.5cm}{0cm}{1cm}{1cm}

\pagenumbering{arabic}


\begin{document}
\author{Kevin O'Brien}
\title{Alternative Graphical Approaches}
\date{\today}
\maketitle
\tableofcontents \setcounter{tocdepth}{2}

\newpage

\section{The Identity plot} This is a simple regression based approach. It gives the analyst a cursory examination of how well the measurement methods agree. In the case of good agreement, the covariates of the plot accord closely with the $X=Y$ line.

\section{Variations of the BA plot}

In light of some potential pitfalls associated with the conventional BA plot, a series of alternative formulations for the Bland-Altman plot have been proposed.



\section{Survival Agreement Plot (Luiz et al)}
This approach is put forward by \citet{Luiz}. It seeks to extend the agreement evaluation through a graphic approach using step functions' capable of expressing the degree of agreement (or disagreement) as a function of several limits of tolerance.

\begin{itemize}
\item It expresses agreement or disagreement as a function of several
limits of tolerance.
\item Y axis represents the proportion of discordant cases.
\item X axis represents the observed differences.
\end{itemize}

\section{Eksborg's Plot}
\citet{Eksborg} proposes a plot of the relative values found by the two
Methods being compared (Method 1/Method 2) vs the mean of the Method
values.

This approach was discussed as an alternative to the BA approach by 
%%%%%%%%%%%%%%%%%%%%%%%%%%%%%%%%%%%%%%%%%%%%%%%%%%%%%%%%%%%%%%%%%%%%%%%%%
\newpage
\section{Bartko's Ellipse}
As an enhancement on the Bland Altman Plot, \citep{bartko} has
expounded a confidence ellipse for the covariates. \citet{bartko} proposes
a bivariate confidence ellipse as a boundary for dispersion. The stated purpose is to 'amplify dispersion', which presumably is for  the purposes of outlier detection.The orientation of the the ellipse is key to interpreting the results.
\begin{itemize}
 \item The Minor Axis is related to the Variance between-subjects
 \item The Major Axis is related to the Error Mean Square.
\end{itemize}
The ellipse illustrates the size of both relative to each
other. Furthermore, the ellipse provides a visual aid to determining the relationship
between the variance of the means $Var(a_{i})$ and the variance of the differences $Var(d_{i})$.
\begin{itemize}
 \item If $Var(a_{i})$ is greater than $Var(d_{i})$, the orientation of the ellipse is horizontal.
 \item If $Var(a_{i})$ is less than $Var(d_{i})$, the orientation of the ellipse is vertical.
\end{itemize}
The more horizontal the ellipse, the greater the ICC.

\newpage

\section{Mountain Plot} Krouwer and Monti have proposed a folded empirical cumulative distribution plot, otherwise known as a Mountain plot.

They argue that it is suitable for detecting large, infrequent errors. This is a non-parametric method that can be used as a complement with the Bland Altman plot.  Mountain plots are created by computing a percentile
for each ranked difference between a new method and a reference method. (Folded plots are so called because of the following transformation is performed for all percentiles above 50: percentile = 100 - percentile.) These percentiles are then plotted against the differences between the two methods.

Krouwer and Monti argue that the mountain plot offers some following advantages. It is easier to find the central $95\%$ of the data, even when the data are not normally distributed. Also, comparison on different distributions can be performed with ease.

\emph{
A mountain plot (or "folded empirical cumulative distribution plot") is created by computing a percentile for each ranked difference between a new method and a reference method. To get a folded plot, the following transformation is performed for all percentiles above 50: percentile = 100 - percentile. These percentiles are then plotted against the differences between the two methods (Krouwer and Monti, 1995).}

\emph{
The mountain plot is a useful complementary plot to the Bland and Altman plot. In particular, the mountain plot offers the following advantages:}
\emph{
It is easier to find the central $95\%$ of the data, even when the data are not Normally distributed.
Different distributions can be compared more easily.
}

%----------------------------------------------------------------%
\section{dewitte et al }
Bland ALtman recommend the logarithmic y scale
others prefer the  precent y scale.
generally there is not much difference(except when the data extends over several orders of magnitude)
percent method is recommends becuase the numbers can be read directly from the plot without the need for back transfromation.

absolute - small range
percentage - medium range
log scale - large range

we observe increasing use of the bland altman plot over the years, from 8% in 1995 to 14% in 1996 and 31%to36% in more recent years.





\addcontentsline{toc}{section}{Bibliography}

\bibliography{2012bib}

\end{document} 
