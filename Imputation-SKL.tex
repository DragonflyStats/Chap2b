mputation of missing values

For various reasons, many real world datasets contain missing values, often encoded as blanks, NaNs or other placeholders. Such datasets however are incompatible with scikit-learn estimators which assume that all values in an array are numerical, and that all have and hold meaning. A basic strategy to use incomplete datasets is to discard entire rows and/or columns containing missing values. However, this comes at the price of losing data which may be valuable (even though incomplete). A better strategy is to impute the missing values, i.e., to infer them from the known part of the data.
The Imputer class provides basic strategies for imputing missing values, either using the mean, the median or the most frequent value of the row or column in which the missing values are located. This class also allows for different missing values encodings.
